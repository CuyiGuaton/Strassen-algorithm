\documentclass[12pt,letterpaper]{scrartcl}
\usepackage{lipsum}
\usepackage[utf8]{inputenc}
\usepackage{amsmath}
\usepackage{amsfonts}
\usepackage{amssymb}
\usepackage{graphicx}
\usepackage[left=3cm,right=2.5cm,top=2.5cm,bottom=2.5cm]{geometry}
\usepackage[]{algorithm2e}
\author{Don cuyi}

\newcommand{\A}[1]{A_{#1}}
%\cof{11}{\alpha}{0}{\alpha}{-\beta}

%Color
\usepackage{color}
\definecolor{nred}{RGB}{174,49,54}
\definecolor{nblue}{RGB}{86,99,146}
\definecolor{nalgo}{RGB}{188,139,76}
\usepackage{sectsty}
\sectionfont{\color{nred}}
\subsectionfont{\color{nblue}}
\subsubsectionfont{\color{nalgo}}

%Librías tikz
\usepackage{pgf,tikz}
\usepackage{mathrsfs}
\usetikzlibrary{arrows}
\usetikzlibrary[patterns]
\newcommand{\degre}{\ensuremath{^\circ}}
\definecolor{qqwuqq}{rgb}{0.,0.39215686274509803,0.}
\definecolor{ffttww}{rgb}{1.,0.2,0.4}
%Hipervinculos
\usepackage{hyperref}

\usepackage{fancyhdr}
\pagestyle{fancy}
\fancyhf{}
\fancyhead[L]{}
\fancyhead[C]{Licenciatura en ciencia de la computación}
\fancyhead[R]{USACH}

%interlineado
\renewcommand{\baselinestretch}{1.2}

%\bibitem{Yahoo} \textsc{Andres G} (2009),
%\textbf{¿Generar números aleatorios negativos en Lenguaje C?} En \textsc{Yahoo! respuestas}
%Recuperado el el 23 del julio del 2014
%\url{https://es.answers.yahoo.com/question/index?qid=20091121055249AAUQH3N}

\newcommand{\biblio}[7]{
\bibitem{#1} \textsc{#2} (#3),
\textbf{#4} En \textsc{#5}
Recuperado el #6
\url{#7}
}

% Last, F. M. (Year Published) Book. City, State: Publisher.
\newcommand{\book}[5]{
\bibitem{#1} \textsc{#2} (#3),
\textbf{#4}  \textsc{#5} Estado: Publicado
}

\begin{document}

\begin{titlepage}

\begin{center}

{\Large { Licenciatura en ciencia de la computación} }

\includegraphics[scale=1]{UDSCNRJ}
\\[1cm]

{\Huge \textsc{Ciclo minimo y Ciclo de Hamilton}}\\[0.7cm]

{\huge  Matemática Computacional}\\[2cm]


\begin{minipage}[l]{0.4\textwidth}
	\begin{flushleft}
	\linespread{1}
		\textbf{\textsf{Profesor:}}\\
		\large Nicolas Thériault
	\end{flushleft}
\end{minipage}
\begin{minipage}[l]{0.4\textwidth}

	\begin{flushright}

		\textbf{\textsf{Autor:}}\\
		\linespread{1}
		\large Sergio Salinas\\
		\large Danilo Abellá\\

	\end{flushright}
\end{minipage}

\end{center}

\end{titlepage}



\newpage

\tableofcontents

\newpage
\section{Introducción}

\section{Análisis teorico}

Dada la multiplicación de dos matrices  A y B
$$
AB =
\begin{pmatrix}
A_{11} & A_{12} \\
A_{21} & A_{22}\\
\end{pmatrix}
\begin{pmatrix}
B_{11} & B_{12} \\
B_{21} & B_{22}\\
\end{pmatrix}
 = \begin{pmatrix}
A_{11} B_{11} + A_{12} B_{21} & A_{11} B_{12} + A_{12} B_{22}\\
A_{21} B_{11} + A_{22} B_{21} & A_{21} B_{12} + A_{22} B_{22}\\
\end{pmatrix}
=
\begin{pmatrix}
C_{11} & C_{12} \\
C_{21} & C_{22}\\
\end{pmatrix}
$$

Se calculan los S y los T de la técnica de Strassen.


\[
\begin{array}{lcl}
	S_1 &=& A_{21} + A_{22} \\
	S_2 &=& A_{21} + A_{22} - A_{11} \\
	S_3 &=& A_{11} - A_{21}\\
	S_4 &=& A_{12} -A_{21} - A_{22} + A_{11} \\
\end{array}
\]

\[
\begin{array}{lcl}
	T_1 &=& B_12 - B_{11}\\
	T_2 &=& B_{22} - B_{12} + B_{11}\\
	T_3 &=& B_{22} - B_{12}\\
	T_4 &=& B_{22} - B_{12} + B_{11} - B_{21}\\
\end{array}
\]
Se calculan los P.
\[
\begin{array}{lcl}
	P_1 &=& A_{11}B_{21}\\
	P_2 &=& A_{12}B_{21}\\
	P_3 &=& A_{12}B_{22} - A_{21}B_{22} - A_{22}B_{22} + A_{11}B_{22}\\
	P_4 &=& A_{22}B_{22} - A_{22}B_{12} + A_{22}B_{11}  - A_{22}B_{21}\\
	P_5 &=& A_{21}B_{11} - A_{21}B_{11} + A_{22}B_{12}- A_{22}B_{11}\\
	P_6 &=& A_{21}B_{22} - A_{21}B_{12} + A_{21}B_{11} + A_{22}B_{22} - A_{22}B_{12} + A_{22}B_{11} - A_{11}B_{22}+ A_{11}B_{22}\\
	&&  + A_{11}B_{12} - A_{11}B_{11}\\
	P_7 &=& A_{11}B_{22} - A_{11}B_{12} - A_{21}B_{22} + A_{21}B_{12}\\
\end{array}
\]
Se calculan los U
\[
\begin{array}{lcl}
	U_{1} &=& A_{11}B_{11} + A_{12}B_{21}\\
	U_{2} &=& A_{21}B_{22} - A_{21}B_{12} + A_{21}B_{11} + A_{22}B_{22} - A_{22}B_{12} + A_{22}B_{11} - A_{11}B_{22} + A_{11}B_{12}\\
	U_{3} &=& A_{21}B_{11} + A_{22}B_{22} - A_{22}B_{12} + A_{22}B_{11}\\
	U_{4} &=& A_{21}B_{22} + A_{22}B_{22} - A_{11}B_{22} + A_{11}B_{12}\\
	U_{5} &=& A_{11}B_{12} + A_{12}B_{22}\\
	U_{6} &=& A_{21}B_{11} + A_{22}B_{21}\\
	U_{7} &=& A_{22}B_{22} + A_{21}B_{12}\\
\end{array}
\]

De está forma se obtiene que

$$
\begin{pmatrix}
	U_1 & U_5\\
	U_6 & U_7\\
\end{pmatrix}
 = \begin{pmatrix}
A_{11} B_{11} + A_{12} B_{21} & A_{11} B_{12} + A_{12} B_{22}\\
A_{21} B_{11} + A_{22} B_{21} & A_{21} B_{12} + A_{22} B_{22}\\
\end{pmatrix}
=
\begin{pmatrix}
C_{11} & C_{12} \\
C_{21} & C_{22}\\
\end{pmatrix}
$$

Con  lo que se demuestra que el algoritmo de strassen es equivalente a la multiplicación de matrices.
\section{Explicación algoritmo}


\newpage

\section{Formulación experimentos}


\newpage

\section{Información de Hardware y Software}


\subsection{ Notebook - Danilo Abellá}
\subsubsection{Software}
\begin{itemize}
\item SO: Xubuntu 16.04.1 LTS
\item GMP Library
\item Mousepad 0.4.0
\end{itemize}

\subsubsection{Hardware}
\begin{itemize}
\item AMD Turion(tm) X2 Dual-Core Mobile RM-72 2.10GHz
\item Memoria (RAM): 4,00 GB(3,75 GB utilizable)
\item Adaptador de pantalla: ATI Raedon HD 3200 Graphics
\end{itemize}



\subsection{Notebook - Sergio Salinas}
\subsubsection{Software}
\begin{itemize}
\item  SO: ubuntu Gnome 16.04 LTS
\item Compilador: gcc version 5.4.0 20160609
\item Editor de text: Atom
\end{itemize}

\subsubsection{Hardware}
\begin{itemize}
\item Procesador: Intel Core i7-6500U CPU  2.50GHz x 4
\item Video: Intel HD Graphics 520 (Skylake GT2)
\end{itemize}
\newpage



\section{Curvas de desempeño de resultados}


\newpage
\section{Conclusiones}


\end{document}
